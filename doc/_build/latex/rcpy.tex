%% Generated by Sphinx.
\def\sphinxdocclass{report}
\documentclass[letterpaper,10pt,english]{sphinxmanual}
\ifdefined\pdfpxdimen
   \let\sphinxpxdimen\pdfpxdimen\else\newdimen\sphinxpxdimen
\fi \sphinxpxdimen=.75bp\relax

\usepackage[utf8]{inputenc}
\ifdefined\DeclareUnicodeCharacter
 \ifdefined\DeclareUnicodeCharacterAsOptional\else
  \DeclareUnicodeCharacter{00A0}{\nobreakspace}
\fi\fi
\usepackage{cmap}
\usepackage[T1]{fontenc}
\usepackage{amsmath,amssymb,amstext}
\usepackage{babel}
\usepackage{times}
\usepackage[Bjarne]{fncychap}
\usepackage[dontkeepoldnames]{sphinx}

\usepackage{geometry}

% Include hyperref last.
\usepackage{hyperref}
% Fix anchor placement for figures with captions.
\usepackage{hypcap}% it must be loaded after hyperref.
% Set up styles of URL: it should be placed after hyperref.
\urlstyle{same}
\addto\captionsenglish{\renewcommand{\contentsname}{Contents:}}

\addto\captionsenglish{\renewcommand{\figurename}{Fig.}}
\addto\captionsenglish{\renewcommand{\tablename}{Table}}
\addto\captionsenglish{\renewcommand{\literalblockname}{Listing}}

\addto\extrasenglish{\def\pageautorefname{page}}

\setcounter{tocdepth}{1}



\title{rcpy Documentation}
\date{May 21, 2017}
\release{0.2a}
\author{Mauricio de Oliveira}
\newcommand{\sphinxlogo}{\vbox{}}
\renewcommand{\releasename}{Release}
\makeindex

\begin{document}

\maketitle
\sphinxtableofcontents
\phantomsection\label{\detokenize{index::doc}}



\chapter{Introduction}
\label{\detokenize{index:python-3-bindings-for-robotics-cape}}\label{\detokenize{index:introduction}}
This package supports the hardware on the \sphinxhref{http://www.strawsondesign.com/}{Robotics Cape} running on a \sphinxhref{https://beagleboard.org/black}{Beaglebone Black} or a \sphinxhref{https://beagleboard.org/blue}{Beaglebone Blue}.


\section{Installation}
\label{\detokenize{index:installation}}
See \sphinxurl{http://github.com/mcdeoliveira/rcpy\#installation} for installation instructions.


\chapter{Module \sphinxtitleref{rcpy}}
\label{\detokenize{index:module-rcpy}}\label{\detokenize{index:rcpy}}\index{rcpy (module)}
This module is responsible to loading and initializing all hardware
devices associated with the Robotics Cape or Beagelbone Blue. Just type:

\begin{sphinxVerbatim}[commandchars=\\\{\}]
\PYG{k+kn}{import} \PYG{n+nn}{rcpy}
\end{sphinxVerbatim}

to load the module. After loading the Robotics Cape is left at the
state {\hyperref[\detokenize{index:rcpy.PAUSED}]{\sphinxcrossref{\sphinxcode{rcpy.PAUSED}}}}. It will also automatically cleanup after
you exit your program. You can add additional function to be called by
the cleanup routine using {\hyperref[\detokenize{index:rcpy.add_cleanup}]{\sphinxcrossref{\sphinxcode{rcpy.add\_cleanup()}}}}.


\section{Constants}
\label{\detokenize{index:constants}}\index{IDLE (in module rcpy)}

\begin{fulllineitems}
\phantomsection\label{\detokenize{index:rcpy.IDLE}}\pysigline{\sphinxcode{rcpy.}\sphinxbfcode{IDLE}}
\end{fulllineitems}

\index{RUNNING (in module rcpy)}

\begin{fulllineitems}
\phantomsection\label{\detokenize{index:rcpy.RUNNING}}\pysigline{\sphinxcode{rcpy.}\sphinxbfcode{RUNNING}}
\end{fulllineitems}

\index{PAUSED (in module rcpy)}

\begin{fulllineitems}
\phantomsection\label{\detokenize{index:rcpy.PAUSED}}\pysigline{\sphinxcode{rcpy.}\sphinxbfcode{PAUSED}}
\end{fulllineitems}

\index{EXITING (in module rcpy)}

\begin{fulllineitems}
\phantomsection\label{\detokenize{index:rcpy.EXITING}}\pysigline{\sphinxcode{rcpy.}\sphinxbfcode{EXITING}}
\end{fulllineitems}



\section{Low-level functions}
\label{\detokenize{index:low-level-functions}}\index{get\_state() (in module rcpy)}

\begin{fulllineitems}
\phantomsection\label{\detokenize{index:rcpy.get_state}}\pysiglinewithargsret{\sphinxcode{rcpy.}\sphinxbfcode{get\_state}}{}{}
Get the current state, one of {\hyperref[\detokenize{index:rcpy.IDLE}]{\sphinxcrossref{\sphinxcode{rcpy.IDLE}}}}, {\hyperref[\detokenize{index:rcpy.RUNNING}]{\sphinxcrossref{\sphinxcode{rcpy.RUNNING}}}}, {\hyperref[\detokenize{index:rcpy.PAUSED}]{\sphinxcrossref{\sphinxcode{rcpy.PAUSED}}}}, {\hyperref[\detokenize{index:rcpy.EXITING}]{\sphinxcrossref{\sphinxcode{rcpy.EXITING}}}}.

\end{fulllineitems}

\index{set\_state() (in module rcpy)}

\begin{fulllineitems}
\phantomsection\label{\detokenize{index:rcpy.set_state}}\pysiglinewithargsret{\sphinxcode{rcpy.}\sphinxbfcode{set\_state}}{\emph{state}}{}
Set the current state, \sphinxtitleref{state} is one of {\hyperref[\detokenize{index:rcpy.IDLE}]{\sphinxcrossref{\sphinxcode{rcpy.IDLE}}}}, {\hyperref[\detokenize{index:rcpy.RUNNING}]{\sphinxcrossref{\sphinxcode{rcpy.RUNNING}}}}, {\hyperref[\detokenize{index:rcpy.PAUSED}]{\sphinxcrossref{\sphinxcode{rcpy.PAUSED}}}}, {\hyperref[\detokenize{index:rcpy.EXITING}]{\sphinxcrossref{\sphinxcode{rcpy.EXITING}}}}.

\end{fulllineitems}

\index{exit() (in module rcpy)}

\begin{fulllineitems}
\phantomsection\label{\detokenize{index:rcpy.exit}}\pysiglinewithargsret{\sphinxcode{rcpy.}\sphinxbfcode{exit}}{}{}
Set state to {\hyperref[\detokenize{index:rcpy.EXITING}]{\sphinxcrossref{\sphinxcode{rcpy.EXITING}}}}.

\end{fulllineitems}

\index{add\_cleanup() (in module rcpy)}

\begin{fulllineitems}
\phantomsection\label{\detokenize{index:rcpy.add_cleanup}}\pysiglinewithargsret{\sphinxcode{rcpy.}\sphinxbfcode{add\_cleanup}}{\emph{fun}, \emph{pars}}{}~\begin{quote}\begin{description}
\item[{Parameters}] \leavevmode\begin{itemize}
\item {} 
\sphinxstyleliteralstrong{fun} \textendash{} function to call at cleanup

\item {} 
\sphinxstyleliteralstrong{pars} \textendash{} list of positional parameters to pass to function \sphinxtitleref{fun}

\end{itemize}

\end{description}\end{quote}

Add function \sphinxtitleref{fun} and parameters \sphinxtitleref{pars} to the list of cleanup functions.

\end{fulllineitems}



\chapter{Module \sphinxtitleref{rcpy.gpio}}
\label{\detokenize{index:module-rcpy-gpio}}\label{\detokenize{index:rcpy-gpio}}\label{\detokenize{index:module-rcpy.gpio}}\index{rcpy.gpio (module)}
This module provides an interface to the GPIO pins used by the
Robotics Cape. There are low level functions which closely mirror the
ones available in the C library and also Classes that provide a higher
level interface.

For example:

\begin{sphinxVerbatim}[commandchars=\\\{\}]
\PYG{k+kn}{import} \PYG{n+nn}{rcpy}\PYG{n+nn}{.}\PYG{n+nn}{gpio} \PYG{k}{as} \PYG{n+nn}{gpio}
\PYG{n}{pause\PYGZus{}button} \PYG{o}{=} \PYG{n}{gpio}\PYG{o}{.}\PYG{n}{Input}\PYG{p}{(}\PYG{n}{gpio}\PYG{o}{.}\PYG{n}{PAUSE\PYGZus{}BTN}\PYG{p}{)}
\end{sphinxVerbatim}

imports the module and create an {\hyperref[\detokenize{index:rcpy.gpio.Input}]{\sphinxcrossref{\sphinxcode{rcpy.gpio.Input}}}} object
corresponding to the \sphinxstyleemphasis{PAUSE} button on the Robotics Cape. The
command:

\begin{sphinxVerbatim}[commandchars=\\\{\}]
\PYG{k}{if} \PYG{n}{pause\PYGZus{}button}\PYG{o}{.}\PYG{n}{low}\PYG{p}{(}\PYG{p}{)}\PYG{p}{:}
  \PYG{n+nb}{print}\PYG{p}{(}\PYG{l+s+s1}{\PYGZsq{}}\PYG{l+s+s1}{Got \PYGZlt{}PAUSE\PYGZgt{}!}\PYG{l+s+s1}{\PYGZsq{}}\PYG{p}{)}
\end{sphinxVerbatim}

waits forever until the \sphinxstyleemphasis{PAUSE} button on the Robotics Cape is pressed
and:

\begin{sphinxVerbatim}[commandchars=\\\{\}]
\PYG{k}{try}\PYG{p}{:}
    \PYG{k}{if} \PYG{n}{pause\PYGZus{}button}\PYG{o}{.}\PYG{n}{low}\PYG{p}{(}\PYG{n}{timeout} \PYG{o}{=} \PYG{l+m+mi}{2000}\PYG{p}{)}\PYG{p}{:}
        \PYG{n+nb}{print}\PYG{p}{(}\PYG{l+s+s1}{\PYGZsq{}}\PYG{l+s+s1}{Got \PYGZlt{}PAUSE\PYGZgt{}!}\PYG{l+s+s1}{\PYGZsq{}}\PYG{p}{)}
\PYG{k}{except} \PYG{n}{gpio}\PYG{o}{.}\PYG{n}{InputTimeout}\PYG{p}{:}
    \PYG{n+nb}{print}\PYG{p}{(}\PYG{l+s+s1}{\PYGZsq{}}\PYG{l+s+s1}{Timed out!}\PYG{l+s+s1}{\PYGZsq{}}\PYG{p}{)}
\end{sphinxVerbatim}

waits for at most 2000 ms, i.e. 2 s, before giving up.

This module also provides the class {\hyperref[\detokenize{index:rcpy.gpio.InputEvent}]{\sphinxcrossref{\sphinxcode{rcpy.gpio.InputEvent}}}} to
handle input events. For example:

\begin{sphinxVerbatim}[commandchars=\\\{\}]
\PYG{k}{class} \PYG{n+nc}{MyInputEvent}\PYG{p}{(}\PYG{n}{gpio}\PYG{o}{.}\PYG{n}{InputEvent}\PYG{p}{)}\PYG{p}{:}

    \PYG{k}{def} \PYG{n+nf}{action}\PYG{p}{(}\PYG{n+nb+bp}{self}\PYG{p}{,} \PYG{n}{event}\PYG{p}{)}\PYG{p}{:}
        \PYG{n+nb}{print}\PYG{p}{(}\PYG{l+s+s1}{\PYGZsq{}}\PYG{l+s+s1}{Got \PYGZlt{}PAUSE\PYGZgt{}!}\PYG{l+s+s1}{\PYGZsq{}}\PYG{p}{)}
\end{sphinxVerbatim}

defines a class that can be used to print \sphinxcode{Got \textless{}PAUSE\textgreater{}!} every
time an input event happens. To connect this class with the particular
event that the \sphinxstyleemphasis{PAUSE} button is pressed instantiate:

\begin{sphinxVerbatim}[commandchars=\\\{\}]
\PYG{n}{pause\PYGZus{}event} \PYG{o}{=} \PYG{n}{MyInputEvent}\PYG{p}{(}\PYG{n}{pause\PYGZus{}button}\PYG{p}{,} \PYG{n}{gpio}\PYG{o}{.}\PYG{n}{InputEvent}\PYG{o}{.}\PYG{n}{LOW}\PYG{p}{)}
\end{sphinxVerbatim}

which will cause the method \sphinxtitleref{action} of the \sphinxtitleref{MyInputEvent} class be
called every time the state of the \sphinxtitleref{pause\_button} becomes
{\hyperref[\detokenize{index:rcpy.gpio.LOW}]{\sphinxcrossref{\sphinxcode{rcpy.gpio.LOW}}}}. The event handler must be started by
calling:

\begin{sphinxVerbatim}[commandchars=\\\{\}]
\PYG{n}{pause\PYGZus{}event}\PYG{o}{.}\PYG{n}{start}\PYG{p}{(}\PYG{p}{)}
\end{sphinxVerbatim}

and it can be stop by:

\begin{sphinxVerbatim}[commandchars=\\\{\}]
\PYG{n}{pause\PYGZus{}event}\PYG{o}{.}\PYG{n}{stop}\PYG{p}{(}\PYG{p}{)}
\end{sphinxVerbatim}

Alternatively one could have created an input event handler by passing
a function to the argument \sphinxtitleref{target} of
{\hyperref[\detokenize{index:rcpy.gpio.InputEvent}]{\sphinxcrossref{\sphinxcode{rcpy.gpio.InputEvent}}}} as in:

\begin{sphinxVerbatim}[commandchars=\\\{\}]
\PYG{k}{def} \PYG{n+nf}{pause\PYGZus{}action}\PYG{p}{(}\PYG{n+nb}{input}\PYG{p}{,} \PYG{n}{event}\PYG{p}{)}\PYG{p}{:}
    \PYG{k}{if} \PYG{n}{event} \PYG{o}{==} \PYG{n}{gpio}\PYG{o}{.}\PYG{n}{InputEvent}\PYG{o}{.}\PYG{n}{LOW}\PYG{p}{:}
        \PYG{n+nb}{print}\PYG{p}{(}\PYG{l+s+s1}{\PYGZsq{}}\PYG{l+s+s1}{\PYGZlt{}PAUSE\PYGZgt{} went LOW}\PYG{l+s+s1}{\PYGZsq{}}\PYG{p}{)}
    \PYG{k}{elif} \PYG{n}{event} \PYG{o}{==} \PYG{n}{gpio}\PYG{o}{.}\PYG{n}{InputEvent}\PYG{o}{.}\PYG{n}{HIGH}\PYG{p}{:}
        \PYG{n+nb}{print}\PYG{p}{(}\PYG{l+s+s1}{\PYGZsq{}}\PYG{l+s+s1}{\PYGZlt{}PAUSE\PYGZgt{} went HIGH}\PYG{l+s+s1}{\PYGZsq{}}\PYG{p}{)}

\PYG{n}{pause\PYGZus{}event} \PYG{o}{=} \PYG{n}{gpio}\PYG{o}{.}\PYG{n}{InputEvent}\PYG{p}{(}\PYG{n}{pause\PYGZus{}button}\PYG{p}{,}
                              \PYG{n}{gpio}\PYG{o}{.}\PYG{n}{InputEvent}\PYG{o}{.}\PYG{n}{LOW} \PYG{o}{\textbar{}} \PYG{n}{gpio}\PYG{o}{.}\PYG{n}{InputEvent}\PYG{o}{.}\PYG{n}{HIGH}\PYG{p}{,}
                              \PYG{n}{target} \PYG{o}{=} \PYG{n}{pause\PYGZus{}action}\PYG{p}{)}
\end{sphinxVerbatim}

Note that the function \sphinxtitleref{pause\_action} will be called when
\sphinxtitleref{pause\_button} becomes either {\hyperref[\detokenize{index:rcpy.gpio.HIGH}]{\sphinxcrossref{\sphinxcode{rcpy.gpio.HIGH}}}} or
{\hyperref[\detokenize{index:rcpy.gpio.LOW}]{\sphinxcrossref{\sphinxcode{rcpy.gpio.LOW}}}} because the event passed to the the
constructor {\hyperref[\detokenize{index:rcpy.gpio.InputEvent}]{\sphinxcrossref{\sphinxcode{InputEvent}}}} is:

\begin{sphinxVerbatim}[commandchars=\\\{\}]
\PYG{n}{gpio}\PYG{o}{.}\PYG{n}{InputEvent}\PYG{o}{.}\PYG{n}{LOW} \PYG{o}{\textbar{}} \PYG{n}{gpio}\PYG{o}{.}\PYG{n}{InputEvent}\PYG{o}{.}\PYG{n}{HIGH}
\end{sphinxVerbatim}

which is joined by the logical or operator \sphinxtitleref{\textbar{}}. The function
\sphinxtitleref{pause\_action} decides on the type of event by checking the variable
\sphinxtitleref{event}. This event handler should be started and stopped using
{\hyperref[\detokenize{index:rcpy.gpio.InputEvent.start}]{\sphinxcrossref{\sphinxcode{rcpy.gpio.InputEvent.start()}}}} and
{\hyperref[\detokenize{index:rcpy.gpio.InputEvent.stop}]{\sphinxcrossref{\sphinxcode{rcpy.gpio.InputEvent.stop()}}}} as before.

Additional positional or keyword arguments can be passed as in:

\begin{sphinxVerbatim}[commandchars=\\\{\}]
\PYG{k}{def} \PYG{n+nf}{pause\PYGZus{}action\PYGZus{}with\PYGZus{}parameter}\PYG{p}{(}\PYG{n+nb}{input}\PYG{p}{,} \PYG{n}{event}\PYG{p}{,} \PYG{n}{parameter}\PYG{p}{)}\PYG{p}{:}
    \PYG{n+nb}{print}\PYG{p}{(}\PYG{l+s+s1}{\PYGZsq{}}\PYG{l+s+s1}{Got \PYGZlt{}PAUSE\PYGZgt{} with }\PYG{l+s+si}{\PYGZob{}\PYGZcb{}}\PYG{l+s+s1}{!}\PYG{l+s+s1}{\PYGZsq{}}\PYG{o}{.}\PYG{n}{format}\PYG{p}{(}\PYG{n}{parameter}\PYG{p}{)}\PYG{p}{)}

\PYG{n}{pause\PYGZus{}event} \PYG{o}{=} \PYG{n}{gpio}\PYG{o}{.}\PYG{n}{InputEvent}\PYG{p}{(}\PYG{n}{pause\PYGZus{}button}\PYG{p}{,} \PYG{n}{gpio}\PYG{o}{.}\PYG{n}{InputEvent}\PYG{o}{.}\PYG{n}{LOW}\PYG{p}{,}
                              \PYG{n}{target} \PYG{o}{=} \PYG{n}{pause\PYGZus{}action\PYGZus{}with\PYGZus{}parameter}\PYG{p}{,}
                              \PYG{n}{vargs} \PYG{o}{=} \PYG{p}{(}\PYG{l+s+s1}{\PYGZsq{}}\PYG{l+s+s1}{some parameter}\PYG{l+s+s1}{\PYGZsq{}}\PYG{p}{,}\PYG{p}{)}\PYG{p}{)}
\end{sphinxVerbatim}

See also {\hyperref[\detokenize{index:rcpy.button.Button}]{\sphinxcrossref{\sphinxcode{rcpy.button.Button}}}} for a better interface for
working with the Robotics Cape buttons.


\section{Constants}
\label{\detokenize{index:id1}}\index{HIGH (in module rcpy.gpio)}

\begin{fulllineitems}
\phantomsection\label{\detokenize{index:rcpy.gpio.HIGH}}\pysigline{\sphinxcode{rcpy.gpio.}\sphinxbfcode{HIGH}}
Logic high level; equals \sphinxtitleref{1}.

\end{fulllineitems}

\index{LOW (in module rcpy.gpio)}

\begin{fulllineitems}
\phantomsection\label{\detokenize{index:rcpy.gpio.LOW}}\pysigline{\sphinxcode{rcpy.gpio.}\sphinxbfcode{LOW}}
Logic low level; equals \sphinxtitleref{0}.

\end{fulllineitems}

\index{POLL\_TIMEOUT (in module rcpy.gpio)}

\begin{fulllineitems}
\phantomsection\label{\detokenize{index:rcpy.gpio.POLL_TIMEOUT}}\pysigline{\sphinxcode{rcpy.gpio.}\sphinxbfcode{POLL\_TIMEOUT}}
Timeout in ms to be used when polling GPIO input (Default 100ms)

\end{fulllineitems}

\index{DEBOUNCE\_INTERVAL (in module rcpy.gpio)}

\begin{fulllineitems}
\phantomsection\label{\detokenize{index:rcpy.gpio.DEBOUNCE_INTERVAL}}\pysigline{\sphinxcode{rcpy.gpio.}\sphinxbfcode{DEBOUNCE\_INTERVAL}}
Interval in ms to be used for debouncing (Default 0.5ms)

\end{fulllineitems}



\section{Classes}
\label{\detokenize{index:classes}}\index{InputTimeout (class in rcpy.gpio)}

\begin{fulllineitems}
\phantomsection\label{\detokenize{index:rcpy.gpio.InputTimeout}}\pysigline{\sphinxstrong{class }\sphinxcode{rcpy.gpio.}\sphinxbfcode{InputTimeout}}
Exception representing an input timeout event.

\end{fulllineitems}

\index{Input (class in rcpy.gpio)}

\begin{fulllineitems}
\phantomsection\label{\detokenize{index:rcpy.gpio.Input}}\pysiglinewithargsret{\sphinxstrong{class }\sphinxcode{rcpy.gpio.}\sphinxbfcode{Input}}{\emph{pin}}{}~\begin{quote}\begin{description}
\item[{Parameters}] \leavevmode
\sphinxstyleliteralstrong{pin} (\sphinxstyleliteralemphasis{int}) \textendash{} GPIO pin

\end{description}\end{quote}

{\hyperref[\detokenize{index:rcpy.gpio.Input}]{\sphinxcrossref{\sphinxcode{rcpy.gpio.Input}}}} represents one of the GPIO input pins in the Robotics Cape or Beaglebone Blue.
\index{is\_high() (rcpy.gpio.Input method)}

\begin{fulllineitems}
\phantomsection\label{\detokenize{index:rcpy.gpio.Input.is_high}}\pysiglinewithargsret{\sphinxbfcode{is\_high}}{}{}~\begin{quote}\begin{description}
\item[{Returns}] \leavevmode
\sphinxcode{True} if pin is equal to {\hyperref[\detokenize{index:rcpy.gpio.HIGH}]{\sphinxcrossref{\sphinxcode{rcpy.gpio.HIGH}}}} and \sphinxcode{False} if pin is {\hyperref[\detokenize{index:rcpy.gpio.LOW}]{\sphinxcrossref{\sphinxcode{rcpy.gpio.LOW}}}}

\end{description}\end{quote}

\end{fulllineitems}

\index{is\_low() (rcpy.gpio.Input method)}

\begin{fulllineitems}
\phantomsection\label{\detokenize{index:rcpy.gpio.Input.is_low}}\pysiglinewithargsret{\sphinxbfcode{is\_low}}{}{}~\begin{quote}\begin{description}
\item[{Returns}] \leavevmode
\sphinxcode{True} if pin is equal to {\hyperref[\detokenize{index:rcpy.gpio.LOW}]{\sphinxcrossref{\sphinxcode{rcpy.gpio.LOW}}}} and \sphinxcode{False} if pin is {\hyperref[\detokenize{index:rcpy.gpio.HIGH}]{\sphinxcrossref{\sphinxcode{rcpy.gpio.HIGH}}}}

\end{description}\end{quote}

\end{fulllineitems}

\index{high\_or\_low() (rcpy.gpio.Input method)}

\begin{fulllineitems}
\phantomsection\label{\detokenize{index:rcpy.gpio.Input.high_or_low}}\pysiglinewithargsret{\sphinxbfcode{high\_or\_low}}{\emph{debounce = 0}, \emph{timeout = None}}{}~\begin{quote}\begin{description}
\item[{Parameters}] \leavevmode\begin{itemize}
\item {} 
\sphinxstyleliteralstrong{debounce} (\sphinxstyleliteralemphasis{int}) \textendash{} number of times to read input for debouncing (default 0)

\item {} 
\sphinxstyleliteralstrong{timeout} (\sphinxstyleliteralemphasis{int}) \textendash{} timeout in milliseconds (default None)

\end{itemize}

\item[{Raises}] \leavevmode
{\hyperref[\detokenize{index:rcpy.gpio.InputTimeout}]{\sphinxcrossref{\sphinxstyleliteralstrong{rcpy.gpio.InputTimeout}}}} \textendash{} if more than \sphinxtitleref{timeout} ms have elapsed without the input changing

\item[{Returns}] \leavevmode
the new state as {\hyperref[\detokenize{index:rcpy.gpio.HIGH}]{\sphinxcrossref{\sphinxcode{rcpy.gpio.HIGH}}}} or {\hyperref[\detokenize{index:rcpy.gpio.LOW}]{\sphinxcrossref{\sphinxcode{rcpy.gpio.LOW}}}}

\end{description}\end{quote}

Wait for pin to change state.

If \sphinxtitleref{timeout} is not \sphinxcode{None} wait at most \sphinxtitleref{timeout} ms.

If \sphinxtitleref{timeout} is negative wait forever. This call cannot be interrupted.

If \sphinxtitleref{timeout} is \sphinxcode{None} wait forever by repeatedly polling in {\hyperref[\detokenize{index:rcpy.gpio.POLL_TIMEOUT}]{\sphinxcrossref{\sphinxcode{rcpy.gpio.POLL\_TIMEOUT}}}} ms. This call can only be interrupted by calling {\hyperref[\detokenize{index:rcpy.exit}]{\sphinxcrossref{\sphinxcode{rcpy.exit()}}}}.

\end{fulllineitems}

\index{high() (rcpy.gpio.Input method)}

\begin{fulllineitems}
\phantomsection\label{\detokenize{index:rcpy.gpio.Input.high}}\pysiglinewithargsret{\sphinxbfcode{high}}{\emph{debounce = 0}, \emph{timeout = None}}{}~\begin{quote}\begin{description}
\item[{Parameters}] \leavevmode\begin{itemize}
\item {} 
\sphinxstyleliteralstrong{debounce} (\sphinxstyleliteralemphasis{int}) \textendash{} number of times to read input for debouncing (default 0)

\item {} 
\sphinxstyleliteralstrong{timeout} (\sphinxstyleliteralemphasis{int}) \textendash{} timeout in milliseconds (default None)

\end{itemize}

\item[{Raises}] \leavevmode
{\hyperref[\detokenize{index:rcpy.gpio.InputTimeout}]{\sphinxcrossref{\sphinxstyleliteralstrong{rcpy.gpio.InputTimeout}}}} \textendash{} if more than \sphinxtitleref{timeout} ms have elapsed without the input changing

\item[{Returns}] \leavevmode
\sphinxcode{True} if the new state is {\hyperref[\detokenize{index:rcpy.gpio.HIGH}]{\sphinxcrossref{\sphinxcode{rcpy.gpio.HIGH}}}} and \sphinxcode{False} if the new state is {\hyperref[\detokenize{index:rcpy.gpio.LOW}]{\sphinxcrossref{\sphinxcode{rcpy.gpio.LOW}}}}

\end{description}\end{quote}

Wait for pin to change state.

If \sphinxtitleref{timeout} is not \sphinxcode{None} wait at most \sphinxtitleref{timeout} ms.

If \sphinxtitleref{timeout} is negative wait forever. This call cannot be interrupted.

If \sphinxtitleref{timeout} is \sphinxcode{None} wait forever by repeatedly polling in {\hyperref[\detokenize{index:rcpy.gpio.POLL_TIMEOUT}]{\sphinxcrossref{\sphinxcode{rcpy.gpio.POLL\_TIMEOUT}}}} ms. This call can only be interrupted by calling {\hyperref[\detokenize{index:rcpy.exit}]{\sphinxcrossref{\sphinxcode{rcpy.exit()}}}}.

\end{fulllineitems}

\index{low() (rcpy.gpio.Input method)}

\begin{fulllineitems}
\phantomsection\label{\detokenize{index:rcpy.gpio.Input.low}}\pysiglinewithargsret{\sphinxbfcode{low}}{\emph{debounce = 0}, \emph{timeout = None}}{}~\begin{quote}\begin{description}
\item[{Parameters}] \leavevmode\begin{itemize}
\item {} 
\sphinxstyleliteralstrong{debounce} (\sphinxstyleliteralemphasis{int}) \textendash{} number of times to read input for debouncing (default 0)

\item {} 
\sphinxstyleliteralstrong{timeout} (\sphinxstyleliteralemphasis{int}) \textendash{} timeout in milliseconds (default None)

\end{itemize}

\item[{Raises}] \leavevmode
{\hyperref[\detokenize{index:rcpy.gpio.InputTimeout}]{\sphinxcrossref{\sphinxstyleliteralstrong{rcpy.gpio.InputTimeout}}}} \textendash{} if more than \sphinxtitleref{timeout} ms have elapsed without the input changing

\item[{Returns}] \leavevmode
\sphinxcode{True} if the new state is {\hyperref[\detokenize{index:rcpy.gpio.LOW}]{\sphinxcrossref{\sphinxcode{rcpy.gpio.LOW}}}} and \sphinxcode{False} if the new state is {\hyperref[\detokenize{index:rcpy.gpio.HIGH}]{\sphinxcrossref{\sphinxcode{rcpy.gpio.HIGH}}}}

\end{description}\end{quote}

Wait for pin to change state.

If \sphinxtitleref{timeout} is not \sphinxcode{None} wait at most \sphinxtitleref{timeout} ms.

If \sphinxtitleref{timeout} is negative wait forever. This call cannot be interrupted.

If \sphinxtitleref{timeout} is \sphinxcode{None} wait forever by repeatedly polling in {\hyperref[\detokenize{index:rcpy.gpio.POLL_TIMEOUT}]{\sphinxcrossref{\sphinxcode{rcpy.gpio.POLL\_TIMEOUT}}}} ms. This call can only be interrupted by calling {\hyperref[\detokenize{index:rcpy.exit}]{\sphinxcrossref{\sphinxcode{rcpy.exit()}}}}.

\end{fulllineitems}


\end{fulllineitems}

\index{InputEvent (class in rcpy.gpio)}

\begin{fulllineitems}
\phantomsection\label{\detokenize{index:rcpy.gpio.InputEvent}}\pysiglinewithargsret{\sphinxstrong{class }\sphinxcode{rcpy.gpio.}\sphinxbfcode{InputEvent}}{\emph{input}, \emph{event}, \emph{debounce = 0}, \emph{timeout = None}, \emph{target = None}, \emph{vargs = ()}, \emph{kwargs = \{\}}}{}~\begin{quote}\begin{description}
\item[{Bases}] \leavevmode
threading.Thread

\end{description}\end{quote}

{\hyperref[\detokenize{index:rcpy.gpio.InputEvent}]{\sphinxcrossref{\sphinxcode{rcpy.gpio.InputEvent}}}} is an event handler for GPIO input events.
\begin{quote}\begin{description}
\item[{Parameters}] \leavevmode\begin{itemize}
\item {} 
\sphinxstyleliteralstrong{input} (\sphinxstyleliteralemphasis{int}) \textendash{} instance of {\hyperref[\detokenize{index:rcpy.gpio.Input}]{\sphinxcrossref{\sphinxcode{rcpy.gpio.Input}}}}

\item {} 
\sphinxstyleliteralstrong{event} (\sphinxstyleliteralemphasis{int}) \textendash{} either {\hyperref[\detokenize{index:rcpy.gpio.InputEvent.HIGH}]{\sphinxcrossref{\sphinxcode{rcpy.gpio.InputEvent.HIGH}}}} or {\hyperref[\detokenize{index:rcpy.gpio.InputEvent.LOW}]{\sphinxcrossref{\sphinxcode{rcpy.gpio.InputEvent.LOW}}}}

\item {} 
\sphinxstyleliteralstrong{debounce} (\sphinxstyleliteralemphasis{int}) \textendash{} number of times to read input for debouncing (default 0)

\item {} 
\sphinxstyleliteralstrong{timeout} (\sphinxstyleliteralemphasis{int}) \textendash{} timeout in milliseconds (default None)

\item {} 
\sphinxstyleliteralstrong{target} (\sphinxstyleliteralemphasis{int}) \textendash{} callback function to run in case input changes (default None)

\item {} 
\sphinxstyleliteralstrong{vargs} (\sphinxstyleliteralemphasis{int}) \textendash{} positional arguments for function \sphinxtitleref{target} (default ())

\item {} 
\sphinxstyleliteralstrong{kwargs} (\sphinxstyleliteralemphasis{int}) \textendash{} keyword arguments for function \sphinxtitleref{target} (default \{\})

\end{itemize}

\end{description}\end{quote}
\index{InputEvent.LOW (in module rcpy.gpio)}

\begin{fulllineitems}
\phantomsection\label{\detokenize{index:rcpy.gpio.InputEvent.LOW}}\pysigline{\sphinxbfcode{LOW}}
Event representing change to a low logic level.

\end{fulllineitems}

\index{InputEvent.HIGH (in module rcpy.gpio)}

\begin{fulllineitems}
\phantomsection\label{\detokenize{index:rcpy.gpio.InputEvent.HIGH}}\pysigline{\sphinxbfcode{HIGH}}
Event representing change to a high logic level.

\end{fulllineitems}

\index{action() (rcpy.gpio.InputEvent method)}

\begin{fulllineitems}
\phantomsection\label{\detokenize{index:rcpy.gpio.InputEvent.action}}\pysiglinewithargsret{\sphinxbfcode{action}}{\emph{event}, \emph{*vargs}, \emph{**kwargs}}{}~\begin{quote}\begin{description}
\item[{Parameters}] \leavevmode\begin{itemize}
\item {} 
\sphinxstyleliteralstrong{event} \textendash{} either {\hyperref[\detokenize{index:rcpy.gpio.HIGH}]{\sphinxcrossref{\sphinxcode{rcpy.gpio.HIGH}}}} or {\hyperref[\detokenize{index:rcpy.gpio.LOW}]{\sphinxcrossref{\sphinxcode{rcpy.gpio.LOW}}}}

\item {} 
\sphinxstyleliteralstrong{vargs} \textendash{} variable positional arguments

\item {} 
\sphinxstyleliteralstrong{kwargs} \textendash{} variable keyword arguments

\end{itemize}

\end{description}\end{quote}

Action to perform when event is detected.

\end{fulllineitems}

\index{start() (rcpy.gpio.InputEvent method)}

\begin{fulllineitems}
\phantomsection\label{\detokenize{index:rcpy.gpio.InputEvent.start}}\pysiglinewithargsret{\sphinxbfcode{start}}{}{}
Start the input event handler thread.

\end{fulllineitems}

\index{stop() (rcpy.gpio.InputEvent method)}

\begin{fulllineitems}
\phantomsection\label{\detokenize{index:rcpy.gpio.InputEvent.stop}}\pysiglinewithargsret{\sphinxbfcode{stop}}{}{}
Attempt to stop the input event handler thread. Once it has stopped it cannot be restarted.

\end{fulllineitems}


\end{fulllineitems}



\section{Low-level functions}
\label{\detokenize{index:id2}}\index{set() (in module rcpy.gpio)}

\begin{fulllineitems}
\phantomsection\label{\detokenize{index:rcpy.gpio.set}}\pysiglinewithargsret{\sphinxcode{rcpy.gpio.}\sphinxbfcode{set}}{\emph{pin}, \emph{value}}{}~\begin{quote}\begin{description}
\item[{Parameters}] \leavevmode\begin{itemize}
\item {} 
\sphinxstyleliteralstrong{pin} (\sphinxstyleliteralemphasis{int}) \textendash{} GPIO pin

\item {} 
\sphinxstyleliteralstrong{value} (\sphinxstyleliteralemphasis{int}) \textendash{} value to set the pin ({\hyperref[\detokenize{index:rcpy.gpio.HIGH}]{\sphinxcrossref{\sphinxcode{rcpy.gpio.HIGH}}}} or {\hyperref[\detokenize{index:rcpy.gpio.LOW}]{\sphinxcrossref{\sphinxcode{rcpy.gpio.LOW}}}})

\end{itemize}

\item[{Raises}] \leavevmode
\sphinxstyleliteralstrong{rcpy.gpio.error} \textendash{} if it cannot write to \sphinxtitleref{pin}

\end{description}\end{quote}

Set GPIO \sphinxtitleref{pin} to the new \sphinxtitleref{value}.

\end{fulllineitems}

\index{get() (in module rcpy.gpio)}

\begin{fulllineitems}
\phantomsection\label{\detokenize{index:rcpy.gpio.get}}\pysiglinewithargsret{\sphinxcode{rcpy.gpio.}\sphinxbfcode{get}}{\emph{pin}}{}~\begin{quote}\begin{description}
\item[{Parameters}] \leavevmode
\sphinxstyleliteralstrong{pin} (\sphinxstyleliteralemphasis{int}) \textendash{} GPIO pin

\item[{Raises}] \leavevmode
\sphinxstyleliteralstrong{rcpy.gpio.error} \textendash{} if it cannot read from \sphinxtitleref{pin}

\item[{Returns}] \leavevmode
the current value of the GPIO \sphinxtitleref{pin}

\end{description}\end{quote}

This is a non-blocking call.

\end{fulllineitems}

\index{read() (in module rcpy.gpio)}

\begin{fulllineitems}
\phantomsection\label{\detokenize{index:rcpy.gpio.read}}\pysiglinewithargsret{\sphinxcode{rcpy.gpio.}\sphinxbfcode{read}}{\emph{pin}, \emph{timeout = None}}{}~\begin{quote}\begin{description}
\item[{Parameters}] \leavevmode\begin{itemize}
\item {} 
\sphinxstyleliteralstrong{pin} (\sphinxstyleliteralemphasis{int}) \textendash{} GPIO pin

\item {} 
\sphinxstyleliteralstrong{timeout} (\sphinxstyleliteralemphasis{int}) \textendash{} timeout in milliseconds (default None)

\end{itemize}

\item[{Raises}] \leavevmode\begin{itemize}
\item {} 
\sphinxstyleliteralstrong{rcpy.gpio.error} \textendash{} if it cannot read from \sphinxtitleref{pin}

\item {} 
{\hyperref[\detokenize{index:rcpy.gpio.InputTimeout}]{\sphinxcrossref{\sphinxstyleliteralstrong{rcpy.gpio.InputTimeout}}}} \textendash{} if more than \sphinxtitleref{timeout} ms have elapsed without the input changing

\end{itemize}

\item[{Returns}] \leavevmode
the new value of the GPIO \sphinxtitleref{pin}

\end{description}\end{quote}

Wait for value of the GPIO \sphinxtitleref{pin} to change. This is a blocking call.

\end{fulllineitems}



\chapter{Module \sphinxtitleref{rcpy.button}}
\label{\detokenize{index:module-rcpy.button}}\label{\detokenize{index:module-rcpy-button}}\label{\detokenize{index:rcpy-button}}\index{rcpy.button (module)}
This module provides an interface to the \sphinxstyleemphasis{PAUSE} and \sphinxstyleemphasis{MODE} buttons in
the Robotics Cape. The command:

\begin{sphinxVerbatim}[commandchars=\\\{\}]
\PYG{k+kn}{import} \PYG{n+nn}{rcpy}\PYG{n+nn}{.}\PYG{n+nn}{button} \PYG{k}{as} \PYG{n+nn}{button}
\end{sphinxVerbatim}

imports the module. The {\hyperref[\detokenize{index:rcpy-button}]{\sphinxcrossref{\DUrole{std,std-ref}{Module rcpy.button}}}} provides
objects corresponding to the \sphinxstyleemphasis{PAUSE} and \sphinxstyleemphasis{MODE} buttons on the
Robotics Cape. Those are {\hyperref[\detokenize{index:rcpy.button.pause}]{\sphinxcrossref{\sphinxcode{rcpy.button.pause}}}} and
{\hyperref[\detokenize{index:rcpy.button.mode}]{\sphinxcrossref{\sphinxcode{rcpy.button.mode}}}}. For example:

\begin{sphinxVerbatim}[commandchars=\\\{\}]
\PYG{k}{if} \PYG{n}{button}\PYG{o}{.}\PYG{n}{mode}\PYG{o}{.}\PYG{n}{pressed}\PYG{p}{(}\PYG{p}{)}\PYG{p}{:}
    \PYG{n+nb}{print}\PYG{p}{(}\PYG{l+s+s1}{\PYGZsq{}}\PYG{l+s+s1}{\PYGZlt{}MODE\PYGZgt{} pressed!}\PYG{l+s+s1}{\PYGZsq{}}\PYG{p}{)}
\end{sphinxVerbatim}

waits forever until the \sphinxstyleemphasis{MODE} button on the Robotics Cape is pressed and:

\begin{sphinxVerbatim}[commandchars=\\\{\}]
\PYG{k}{if} \PYG{n}{button}\PYG{o}{.}\PYG{n}{mode}\PYG{o}{.}\PYG{n}{released}\PYG{p}{(}\PYG{p}{)}\PYG{p}{:}
    \PYG{n+nb}{print}\PYG{p}{(}\PYG{l+s+s1}{\PYGZsq{}}\PYG{l+s+s1}{\PYGZlt{}MODE\PYGZgt{} released!}\PYG{l+s+s1}{\PYGZsq{}}\PYG{p}{)}
\end{sphinxVerbatim}

waits forever until the \sphinxstyleemphasis{MODE} button on the Robotics Cape is
released. Note that nothing will print if you first have to press the
button before releasing because {\hyperref[\detokenize{index:rcpy.button.Button.released}]{\sphinxcrossref{\sphinxcode{rcpy.button.Button.released()}}}}
returns \sphinxcode{False} after the first input event, which in this case
was \sphinxstyleemphasis{pressed}. As with {\hyperref[\detokenize{index:rcpy-gpio}]{\sphinxcrossref{\DUrole{std,std-ref}{Module rcpy.gpio}}}}, it is possible to use
{\hyperref[\detokenize{index:rcpy.gpio.InputTimeout}]{\sphinxcrossref{\sphinxcode{rcpy.gpio.InputTimeout}}}} as in:

\begin{sphinxVerbatim}[commandchars=\\\{\}]
\PYG{k+kn}{import} \PYG{n+nn}{rcpy}\PYG{n+nn}{.}\PYG{n+nn}{gpio} \PYG{k}{as} \PYG{n+nn}{gpio}
\PYG{k}{try}\PYG{p}{:}
    \PYG{k}{if} \PYG{n}{button}\PYG{o}{.}\PYG{n}{mode}\PYG{o}{.}\PYG{n}{pressed}\PYG{p}{(}\PYG{n}{timeout} \PYG{o}{=} \PYG{l+m+mi}{2000}\PYG{p}{)}\PYG{p}{:}
        \PYG{n+nb}{print}\PYG{p}{(}\PYG{l+s+s1}{\PYGZsq{}}\PYG{l+s+s1}{\PYGZlt{}MODE\PYGZgt{} pressed!}\PYG{l+s+s1}{\PYGZsq{}}\PYG{p}{)}
\PYG{k}{except} \PYG{n}{gpio}\PYG{o}{.}\PYG{n}{InputTimeout}\PYG{p}{:}
    \PYG{n+nb}{print}\PYG{p}{(}\PYG{l+s+s1}{\PYGZsq{}}\PYG{l+s+s1}{Timed out!}\PYG{l+s+s1}{\PYGZsq{}}\PYG{p}{)}
\end{sphinxVerbatim}

which waits for at most 2000 ms, i.e. 2 s, before giving up.

This module also provides the class {\hyperref[\detokenize{index:rcpy.button.ButtonEvent}]{\sphinxcrossref{\sphinxcode{rcpy.button.ButtonEvent}}}} to
handle input events. For example:

\begin{sphinxVerbatim}[commandchars=\\\{\}]
\PYG{k}{class} \PYG{n+nc}{MyButtonEvent}\PYG{p}{(}\PYG{n}{button}\PYG{o}{.}\PYG{n}{ButtonEvent}\PYG{p}{)}\PYG{p}{:}

    \PYG{k}{def} \PYG{n+nf}{action}\PYG{p}{(}\PYG{n+nb+bp}{self}\PYG{p}{,} \PYG{n}{event}\PYG{p}{)}\PYG{p}{:}
        \PYG{n+nb}{print}\PYG{p}{(}\PYG{l+s+s1}{\PYGZsq{}}\PYG{l+s+s1}{Got \PYGZlt{}PAUSE\PYGZgt{}!}\PYG{l+s+s1}{\PYGZsq{}}\PYG{p}{)}
\end{sphinxVerbatim}

defines a class that can be used to print \sphinxcode{Got \textless{}PAUSE\textgreater{}!} every
time the \sphinxstyleemphasis{PAUSE} button is pressed. To instantiate and start the event
handler use:

\begin{sphinxVerbatim}[commandchars=\\\{\}]
\PYG{n}{pause\PYGZus{}event} \PYG{o}{=} \PYG{n}{MyButtonEvent}\PYG{p}{(}\PYG{n}{button}\PYG{o}{.}\PYG{n}{pause}\PYG{p}{,} \PYG{n}{button}\PYG{o}{.}\PYG{n}{ButtonEvent}\PYG{o}{.}\PYG{n}{PRESSED}\PYG{p}{)}
\PYG{n}{pause\PYGZus{}event}\PYG{o}{.}\PYG{n}{start}\PYG{p}{(}\PYG{p}{)}
\end{sphinxVerbatim}

The event handler can be stop by calling:

\begin{sphinxVerbatim}[commandchars=\\\{\}]
\PYG{n}{pause\PYGZus{}event}\PYG{o}{.}\PYG{n}{stop}\PYG{p}{(}\PYG{p}{)}
\end{sphinxVerbatim}

Alternatively one could have created an input event handler by passing
a function to the argument \sphinxtitleref{target} of
{\hyperref[\detokenize{index:rcpy.button.ButtonEvent}]{\sphinxcrossref{\sphinxcode{rcpy.button.ButtonEvent}}}} as in:

\begin{sphinxVerbatim}[commandchars=\\\{\}]
\PYG{k}{def} \PYG{n+nf}{pause\PYGZus{}action}\PYG{p}{(}\PYG{n+nb}{input}\PYG{p}{,} \PYG{n}{event}\PYG{p}{)}\PYG{p}{:}
    \PYG{k}{if} \PYG{n}{event} \PYG{o}{==} \PYG{n}{button}\PYG{o}{.}\PYG{n}{ButtonEvent}\PYG{o}{.}\PYG{n}{PRESSED}\PYG{p}{:}
        \PYG{n+nb}{print}\PYG{p}{(}\PYG{l+s+s1}{\PYGZsq{}}\PYG{l+s+s1}{\PYGZlt{}PAUSE\PYGZgt{} pressed!}\PYG{l+s+s1}{\PYGZsq{}}\PYG{p}{)}
    \PYG{k}{elif} \PYG{n}{event} \PYG{o}{==} \PYG{n}{button}\PYG{o}{.}\PYG{n}{ButtonEvent}\PYG{o}{.}\PYG{n}{RELEASED}\PYG{p}{:}
        \PYG{n+nb}{print}\PYG{p}{(}\PYG{l+s+s1}{\PYGZsq{}}\PYG{l+s+s1}{\PYGZlt{}PAUSE\PYGZgt{} released!}\PYG{l+s+s1}{\PYGZsq{}}\PYG{p}{)}

\PYG{n}{pause\PYGZus{}event} \PYG{o}{=} \PYG{n}{button}\PYG{o}{.}\PYG{n}{ButtonEvent}\PYG{p}{(}\PYG{n}{button}\PYG{o}{.}\PYG{n}{pause}\PYG{p}{,}
                                 \PYG{n}{button}\PYG{o}{.}\PYG{n}{ButtonEvent}\PYG{o}{.}\PYG{n}{PRESSED} \PYG{o}{\textbar{}} \PYG{n}{button}\PYG{o}{.}\PYG{n}{ButtonEvent}\PYG{o}{.}\PYG{n}{RELEASED}\PYG{p}{,}
                                 \PYG{n}{target} \PYG{o}{=} \PYG{n}{pause\PYGZus{}action}\PYG{p}{)}
\end{sphinxVerbatim}

This event handler should be started and stopped using
{\hyperref[\detokenize{index:rcpy.button.ButtonEvent.start}]{\sphinxcrossref{\sphinxcode{rcpy.button.ButtonEvent.start()}}}} and
{\hyperref[\detokenize{index:rcpy.button.ButtonEvent.stop}]{\sphinxcrossref{\sphinxcode{rcpy.button.ButtonEvent.stop()}}}} as in
{\hyperref[\detokenize{index:rcpy-gpio}]{\sphinxcrossref{\DUrole{std,std-ref}{Module rcpy.gpio}}}}. Additional positional or keyword arguments can be
passed as in:

\begin{sphinxVerbatim}[commandchars=\\\{\}]
\PYG{k}{def} \PYG{n+nf}{pause\PYGZus{}action\PYGZus{}with\PYGZus{}parameter}\PYG{p}{(}\PYG{n+nb}{input}\PYG{p}{,} \PYG{n}{event}\PYG{p}{,} \PYG{n}{parameter}\PYG{p}{)}\PYG{p}{:}
    \PYG{n+nb}{print}\PYG{p}{(}\PYG{l+s+s1}{\PYGZsq{}}\PYG{l+s+s1}{Got \PYGZlt{}PAUSE\PYGZgt{} with }\PYG{l+s+si}{\PYGZob{}\PYGZcb{}}\PYG{l+s+s1}{!}\PYG{l+s+s1}{\PYGZsq{}}\PYG{o}{.}\PYG{n}{format}\PYG{p}{(}\PYG{n}{parameter}\PYG{p}{)}\PYG{p}{)}

\PYG{n}{pause\PYGZus{}event} \PYG{o}{=} \PYG{n}{button}\PYG{o}{.}\PYG{n}{ButtonEvent}\PYG{p}{(}\PYG{n}{button}\PYG{o}{.}\PYG{n}{pause}\PYG{p}{,} \PYG{n}{button}\PYG{o}{.}\PYG{n}{ButtonEvent}\PYG{o}{.}\PYG{n}{PRESSED}\PYG{p}{,}
                                 \PYG{n}{target} \PYG{o}{=} \PYG{n}{pause\PYGZus{}action\PYGZus{}with\PYGZus{}parameter}\PYG{p}{,}
                                 \PYG{n}{vargs} \PYG{o}{=} \PYG{p}{(}\PYG{l+s+s1}{\PYGZsq{}}\PYG{l+s+s1}{some parameter}\PYG{l+s+s1}{\PYGZsq{}}\PYG{p}{,}\PYG{p}{)}\PYG{p}{)}
\end{sphinxVerbatim}

The main difference between {\hyperref[\detokenize{index:rcpy-button}]{\sphinxcrossref{\DUrole{std,std-ref}{Module rcpy.button}}}} and {\hyperref[\detokenize{index:rcpy-gpio}]{\sphinxcrossref{\DUrole{std,std-ref}{Module rcpy.gpio}}}} is
that {\hyperref[\detokenize{index:rcpy-button}]{\sphinxcrossref{\DUrole{std,std-ref}{Module rcpy.button}}}} defines the constants
{\hyperref[\detokenize{index:rcpy.button.PRESSED}]{\sphinxcrossref{\sphinxcode{rcpy.button.PRESSED}}}} and {\hyperref[\detokenize{index:rcpy.button.RELEASED}]{\sphinxcrossref{\sphinxcode{rcpy.button.RELEASED}}}},
the events {\hyperref[\detokenize{index:rcpy.button.ButtonEvent.PRESSED}]{\sphinxcrossref{\sphinxcode{rcpy.button.ButtonEvent.PRESSED}}}} and
{\hyperref[\detokenize{index:rcpy.button.ButtonEvent.RELEASED}]{\sphinxcrossref{\sphinxcode{rcpy.button.ButtonEvent.RELEASED}}}}, and its classes handle
debouncing by default.


\section{Constants}
\label{\detokenize{index:id3}}\index{PRESSED (in module rcpy.button)}

\begin{fulllineitems}
\phantomsection\label{\detokenize{index:rcpy.button.PRESSED}}\pysigline{\sphinxcode{rcpy.button.}\sphinxbfcode{PRESSED}}
State of a pressed button; equal to {\hyperref[\detokenize{index:rcpy.gpio.LOW}]{\sphinxcrossref{\sphinxcode{rcpy.gpio.LOW}}}}.

\end{fulllineitems}

\index{RELEASED (in module rcpy.button)}

\begin{fulllineitems}
\phantomsection\label{\detokenize{index:rcpy.button.RELEASED}}\pysigline{\sphinxcode{rcpy.button.}\sphinxbfcode{RELEASED}}
State of a released button; equal to {\hyperref[\detokenize{index:rcpy.gpio.HIGH}]{\sphinxcrossref{\sphinxcode{rcpy.gpio.HIGH}}}}.

\end{fulllineitems}

\index{pause (in module rcpy.button)}

\begin{fulllineitems}
\phantomsection\label{\detokenize{index:rcpy.button.pause}}\pysigline{\sphinxcode{rcpy.button.}\sphinxbfcode{pause}}
{\hyperref[\detokenize{index:rcpy.button.Button}]{\sphinxcrossref{\sphinxcode{rcpy.button.Button}}}} representing the Robotics Cape \sphinxstyleemphasis{PAUSE} button.

\end{fulllineitems}

\index{mode (in module rcpy.button)}

\begin{fulllineitems}
\phantomsection\label{\detokenize{index:rcpy.button.mode}}\pysigline{\sphinxcode{rcpy.button.}\sphinxbfcode{mode}}
{\hyperref[\detokenize{index:rcpy.button.Button}]{\sphinxcrossref{\sphinxcode{rcpy.button.Button}}}} representing the Robotics Cape \sphinxstyleemphasis{MODE} button.

\end{fulllineitems}

\index{DEBOUNCE (in module rcpy.button)}

\begin{fulllineitems}
\phantomsection\label{\detokenize{index:rcpy.button.DEBOUNCE}}\pysigline{\sphinxcode{rcpy.button.}\sphinxbfcode{DEBOUNCE}}
Number of times to test for deboucing (Default 3)

\end{fulllineitems}



\section{Classes}
\label{\detokenize{index:id4}}\index{Button (class in rcpy.button)}

\begin{fulllineitems}
\phantomsection\label{\detokenize{index:rcpy.button.Button}}\pysigline{\sphinxstrong{class }\sphinxcode{rcpy.button.}\sphinxbfcode{Button}}~\begin{quote}\begin{description}
\item[{Bases}] \leavevmode
{\hyperref[\detokenize{index:rcpy.gpio.Input}]{\sphinxcrossref{\sphinxcode{rcpy.gpio.Input}}}}

\end{description}\end{quote}

{\hyperref[\detokenize{index:rcpy.button.Button}]{\sphinxcrossref{\sphinxcode{rcpy.button.Button}}}} represents buttons in the Robotics Cape or Beaglebone Blue.
\index{is\_pressed() (rcpy.button.Button method)}

\begin{fulllineitems}
\phantomsection\label{\detokenize{index:rcpy.button.Button.is_pressed}}\pysiglinewithargsret{\sphinxbfcode{is\_pressed}}{\emph{debounce = rcpy.button.DEBOUNCE}, \emph{timeout = None}}{}~\begin{quote}\begin{description}
\item[{Returns}] \leavevmode
\sphinxcode{True} if button state is equal to \sphinxcode{rcpy.gpio.PRESSED} and \sphinxcode{False} if pin is \sphinxcode{rcpy.gpio.RELEASED}

\end{description}\end{quote}

\end{fulllineitems}

\index{is\_released() (rcpy.button.Button method)}

\begin{fulllineitems}
\phantomsection\label{\detokenize{index:rcpy.button.Button.is_released}}\pysiglinewithargsret{\sphinxbfcode{is\_released}}{\emph{debounce = rcpy.button.DEBOUNCE}, \emph{timeout = None}}{}~\begin{quote}\begin{description}
\item[{Returns}] \leavevmode
\sphinxcode{True} if button state is equal to \sphinxcode{rcpy.gpio.RELEASED} and \sphinxcode{False} if pin is \sphinxcode{rcpy.gpio.PRESSED}

\end{description}\end{quote}

\end{fulllineitems}

\index{pressed\_or\_released() (rcpy.button.Button method)}

\begin{fulllineitems}
\phantomsection\label{\detokenize{index:rcpy.button.Button.pressed_or_released}}\pysiglinewithargsret{\sphinxbfcode{pressed\_or\_released}}{\emph{debounce = rcpy.button.DEBOUNCE}, \emph{timeout = None}}{}~\begin{quote}\begin{description}
\item[{Parameters}] \leavevmode\begin{itemize}
\item {} 
\sphinxstyleliteralstrong{debounce} (\sphinxstyleliteralemphasis{int}) \textendash{} number of times to read input for debouncing (default \sphinxtitleref{rcpy.button.DEBOUNCE})

\item {} 
\sphinxstyleliteralstrong{timeout} (\sphinxstyleliteralemphasis{int}) \textendash{} timeout in milliseconds (default None)

\end{itemize}

\item[{Raises}] \leavevmode
{\hyperref[\detokenize{index:rcpy.gpio.InputTimeout}]{\sphinxcrossref{\sphinxstyleliteralstrong{rcpy.gpio.InputTimeout}}}} \textendash{} if more than \sphinxtitleref{timeout} ms have elapsed without the button state changing

\item[{Returns}] \leavevmode
the new state as {\hyperref[\detokenize{index:rcpy.button.PRESSED}]{\sphinxcrossref{\sphinxcode{rcpy.button.PRESSED}}}} or {\hyperref[\detokenize{index:rcpy.button.RELEASED}]{\sphinxcrossref{\sphinxcode{rcpy.button.RELEASED}}}}

\end{description}\end{quote}

Wait for button state to change.

If \sphinxtitleref{timeout} is not \sphinxcode{None} wait at most \sphinxtitleref{timeout} ms.

If \sphinxtitleref{timeout} is negative wait forever. This call cannot be interrupted.

If \sphinxtitleref{timeout} is \sphinxcode{None} wait forever by repeatedly polling in {\hyperref[\detokenize{index:rcpy.gpio.POLL_TIMEOUT}]{\sphinxcrossref{\sphinxcode{rcpy.gpio.POLL\_TIMEOUT}}}} ms. This call can only be interrupted by calling {\hyperref[\detokenize{index:rcpy.exit}]{\sphinxcrossref{\sphinxcode{rcpy.exit()}}}}.

\end{fulllineitems}

\index{pressed() (rcpy.button.Button method)}

\begin{fulllineitems}
\phantomsection\label{\detokenize{index:rcpy.button.Button.pressed}}\pysiglinewithargsret{\sphinxbfcode{pressed}}{\emph{debounce = rcpy.button.DEBOUNCE}, \emph{timeout = None}}{}~\begin{quote}\begin{description}
\item[{Parameters}] \leavevmode\begin{itemize}
\item {} 
\sphinxstyleliteralstrong{debounce} (\sphinxstyleliteralemphasis{int}) \textendash{} number of times to read input for debouncing (default \sphinxtitleref{rcpy.button.DEBOUNCE})

\item {} 
\sphinxstyleliteralstrong{timeout} (\sphinxstyleliteralemphasis{int}) \textendash{} timeout in milliseconds (default None)

\end{itemize}

\item[{Raises}] \leavevmode
{\hyperref[\detokenize{index:rcpy.gpio.InputTimeout}]{\sphinxcrossref{\sphinxstyleliteralstrong{rcpy.gpio.InputTimeout}}}} \textendash{} if more than \sphinxtitleref{timeout} ms have elapsed without the button state changing.

\item[{Returns}] \leavevmode
\sphinxcode{True} if the new state is {\hyperref[\detokenize{index:rcpy.button.PRESSED}]{\sphinxcrossref{\sphinxcode{rcpy.button.PRESSED}}}} and \sphinxcode{False} if the new state is {\hyperref[\detokenize{index:rcpy.button.RELEASED}]{\sphinxcrossref{\sphinxcode{rcpy.button.RELEASED}}}}

\end{description}\end{quote}

Wait for button state to change.

If \sphinxtitleref{timeout} is not \sphinxcode{None} wait at most \sphinxtitleref{timeout} ms.

If \sphinxtitleref{timeout} is negative wait forever. This call cannot be interrupted.

If \sphinxtitleref{timeout} is \sphinxcode{None} wait forever by repeatedly polling in {\hyperref[\detokenize{index:rcpy.gpio.POLL_TIMEOUT}]{\sphinxcrossref{\sphinxcode{rcpy.gpio.POLL\_TIMEOUT}}}} ms. This call can only be interrupted by calling {\hyperref[\detokenize{index:rcpy.exit}]{\sphinxcrossref{\sphinxcode{rcpy.exit()}}}}.

\end{fulllineitems}

\index{released() (rcpy.button.Button method)}

\begin{fulllineitems}
\phantomsection\label{\detokenize{index:rcpy.button.Button.released}}\pysiglinewithargsret{\sphinxbfcode{released}}{\emph{debounce = rcpy.button.DEBOUNCE}, \emph{timeout = None}}{}~\begin{quote}\begin{description}
\item[{Parameters}] \leavevmode\begin{itemize}
\item {} 
\sphinxstyleliteralstrong{debounce} (\sphinxstyleliteralemphasis{int}) \textendash{} number of times to read input for debouncing (default \sphinxtitleref{rcpy.button.DEBOUNCE})

\item {} 
\sphinxstyleliteralstrong{timeout} (\sphinxstyleliteralemphasis{int}) \textendash{} timeout in milliseconds (default None)

\end{itemize}

\item[{Raises}] \leavevmode
{\hyperref[\detokenize{index:rcpy.gpio.InputTimeout}]{\sphinxcrossref{\sphinxstyleliteralstrong{rcpy.gpio.InputTimeout}}}} \textendash{} if more than \sphinxtitleref{timeout} ms have elapsed without the button state changing.

\item[{Returns}] \leavevmode
\sphinxcode{True} if the new state is {\hyperref[\detokenize{index:rcpy.button.RELEASED}]{\sphinxcrossref{\sphinxcode{rcpy.button.RELEASED}}}} and \sphinxcode{False} if the new state is {\hyperref[\detokenize{index:rcpy.button.PRESSED}]{\sphinxcrossref{\sphinxcode{rcpy.button.PRESSED}}}}

\end{description}\end{quote}

Wait for button state to change.

If \sphinxtitleref{timeout} is not \sphinxcode{None} wait at most \sphinxtitleref{timeout} ms.

If \sphinxtitleref{timeout} is negative wait forever. This call cannot be interrupted.

If \sphinxtitleref{timeout} is \sphinxcode{None} wait forever by repeatedly polling in {\hyperref[\detokenize{index:rcpy.gpio.POLL_TIMEOUT}]{\sphinxcrossref{\sphinxcode{rcpy.gpio.POLL\_TIMEOUT}}}} ms. This call can only be interrupted by calling {\hyperref[\detokenize{index:rcpy.exit}]{\sphinxcrossref{\sphinxcode{rcpy.exit()}}}}.

\end{fulllineitems}


\end{fulllineitems}

\index{ButtonEvent (class in rcpy.button)}

\begin{fulllineitems}
\phantomsection\label{\detokenize{index:rcpy.button.ButtonEvent}}\pysiglinewithargsret{\sphinxstrong{class }\sphinxcode{rcpy.button.}\sphinxbfcode{ButtonEvent}}{\emph{input}, \emph{event}, \emph{debounce = rcpy.button.DEBOUNCE}, \emph{timeout = None}, \emph{target = None}, \emph{vargs = ()}, \emph{kwargs = \{\}}}{}~\begin{quote}\begin{description}
\item[{Bases}] \leavevmode
{\hyperref[\detokenize{index:rcpy.gpio.InputEvent}]{\sphinxcrossref{\sphinxcode{rcpy.gpio.InputEvent}}}}

\item[{Parameters}] \leavevmode\begin{itemize}
\item {} 
\sphinxstyleliteralstrong{input} (\sphinxstyleliteralemphasis{int}) \textendash{} instance of {\hyperref[\detokenize{index:rcpy.gpio.Input}]{\sphinxcrossref{\sphinxcode{rcpy.gpio.Input}}}}

\item {} 
\sphinxstyleliteralstrong{event} (\sphinxstyleliteralemphasis{int}) \textendash{} either {\hyperref[\detokenize{index:rcpy.button.ButtonEvent.PRESSED}]{\sphinxcrossref{\sphinxcode{rcpy.button.ButtonEvent.PRESSED}}}} or {\hyperref[\detokenize{index:rcpy.button.ButtonEvent.RELEASED}]{\sphinxcrossref{\sphinxcode{rcpy.button.ButtonEvent.RELEASED}}}}

\item {} 
\sphinxstyleliteralstrong{debounce} (\sphinxstyleliteralemphasis{int}) \textendash{} number of times to read input for debouncing (default \sphinxtitleref{rcpy.button.DEBOUNCE})

\item {} 
\sphinxstyleliteralstrong{timeout} (\sphinxstyleliteralemphasis{int}) \textendash{} timeout in milliseconds (default \sphinxtitleref{None})

\item {} 
\sphinxstyleliteralstrong{target} (\sphinxstyleliteralemphasis{int}) \textendash{} callback function to run in case input changes (default \sphinxtitleref{None})

\item {} 
\sphinxstyleliteralstrong{vargs} (\sphinxstyleliteralemphasis{int}) \textendash{} positional arguments for function \sphinxtitleref{target} (default \sphinxtitleref{()})

\item {} 
\sphinxstyleliteralstrong{kwargs} (\sphinxstyleliteralemphasis{int}) \textendash{} keyword arguments for function \sphinxtitleref{target} (default \sphinxtitleref{\{\}})

\end{itemize}

\end{description}\end{quote}

{\hyperref[\detokenize{index:rcpy.button.ButtonEvent}]{\sphinxcrossref{\sphinxcode{rcpy.button.ButtonEvent}}}} is an event handler for button events.
\index{ButtonEvent.PRESSED (in module rcpy.button)}

\begin{fulllineitems}
\phantomsection\label{\detokenize{index:rcpy.button.ButtonEvent.PRESSED}}\pysigline{\sphinxbfcode{PRESSED}}
Event representing pressing a button; equal to {\hyperref[\detokenize{index:rcpy.gpio.InputEvent.LOW}]{\sphinxcrossref{\sphinxcode{rcpy.gpio.InputEvent.LOW}}}}.

\end{fulllineitems}

\index{ButtonEvent.RELEASED (in module rcpy.button)}

\begin{fulllineitems}
\phantomsection\label{\detokenize{index:rcpy.button.ButtonEvent.RELEASED}}\pysigline{\sphinxbfcode{RELEASED}}
Event representing releasing a button; equal to {\hyperref[\detokenize{index:rcpy.gpio.InputEvent.HIGH}]{\sphinxcrossref{\sphinxcode{rcpy.gpio.InputEvent.HIGH}}}}.

\end{fulllineitems}

\index{action() (rcpy.button.ButtonEvent method)}

\begin{fulllineitems}
\phantomsection\label{\detokenize{index:rcpy.button.ButtonEvent.action}}\pysiglinewithargsret{\sphinxbfcode{action}}{\emph{event}, \emph{*vargs}, \emph{**kwargs}}{}~\begin{quote}\begin{description}
\item[{Parameters}] \leavevmode\begin{itemize}
\item {} 
\sphinxstyleliteralstrong{event} \textendash{} either {\hyperref[\detokenize{index:rcpy.button.PRESSED}]{\sphinxcrossref{\sphinxcode{rcpy.button.PRESSED}}}} or {\hyperref[\detokenize{index:rcpy.button.RELEASED}]{\sphinxcrossref{\sphinxcode{rcpy.button.RELEASED}}}}

\item {} 
\sphinxstyleliteralstrong{vargs} \textendash{} variable positional arguments

\item {} 
\sphinxstyleliteralstrong{kwargs} \textendash{} variable keyword arguments

\end{itemize}

\end{description}\end{quote}

Action to perform when event is detected.

\end{fulllineitems}

\index{start() (rcpy.button.ButtonEvent method)}

\begin{fulllineitems}
\phantomsection\label{\detokenize{index:rcpy.button.ButtonEvent.start}}\pysiglinewithargsret{\sphinxbfcode{start}}{}{}
Start the input event handler thread.

\end{fulllineitems}

\index{stop() (rcpy.button.ButtonEvent method)}

\begin{fulllineitems}
\phantomsection\label{\detokenize{index:rcpy.button.ButtonEvent.stop}}\pysiglinewithargsret{\sphinxbfcode{stop}}{}{}
Attempt to stop the input event handler thread. Once it has stopped it cannot be restarted.

\end{fulllineitems}


\end{fulllineitems}



\chapter{Module \sphinxtitleref{rcpy.led}}
\label{\detokenize{index:rcpy-led}}\label{\detokenize{index:module-rcpy-led}}\label{\detokenize{index:module-rcpy.led}}\index{rcpy.led (module)}
This module provides an interface to the \sphinxstyleemphasis{RED} and \sphinxstyleemphasis{GREEN} buttons in
the Robotics Cape. The command:

\begin{sphinxVerbatim}[commandchars=\\\{\}]
\PYG{k+kn}{import} \PYG{n+nn}{rcpy}\PYG{n+nn}{.}\PYG{n+nn}{led} \PYG{k}{as} \PYG{n+nn}{led}
\end{sphinxVerbatim}

imports the module. The {\hyperref[\detokenize{index:rcpy-led}]{\sphinxcrossref{\DUrole{std,std-ref}{Module rcpy.led}}}} provides objects corresponding
to the \sphinxstyleemphasis{RED} and \sphinxstyleemphasis{GREEN} buttons on the Robotics Cape, namely
{\hyperref[\detokenize{index:rcpy.led.red}]{\sphinxcrossref{\sphinxcode{rcpy.led.red}}}} and {\hyperref[\detokenize{index:rcpy.led.green}]{\sphinxcrossref{\sphinxcode{rcpy.led.green}}}}. For example:

\begin{sphinxVerbatim}[commandchars=\\\{\}]
\PYG{n}{led}\PYG{o}{.}\PYG{n}{red}\PYG{o}{.}\PYG{n}{on}\PYG{p}{(}\PYG{p}{)}
\end{sphinxVerbatim}

turns the \sphinxstyleemphasis{RED} LED on and:

\begin{sphinxVerbatim}[commandchars=\\\{\}]
\PYG{n}{led}\PYG{o}{.}\PYG{n}{green}\PYG{o}{.}\PYG{n}{off}\PYG{p}{(}\PYG{p}{)}
\end{sphinxVerbatim}

turns the \sphinxstyleemphasis{GREEN} LED off. Likewise:

\begin{sphinxVerbatim}[commandchars=\\\{\}]
\PYG{n}{led}\PYG{o}{.}\PYG{n}{green}\PYG{o}{.}\PYG{n}{is\PYGZus{}on}\PYG{p}{(}\PYG{p}{)}
\end{sphinxVerbatim}

returns \sphinxcode{True} if the \sphinxstyleemphasis{GREEN} LED is on and:

\begin{sphinxVerbatim}[commandchars=\\\{\}]
\PYG{n}{led}\PYG{o}{.}\PYG{n}{red}\PYG{o}{.}\PYG{n}{is\PYGZus{}off}\PYG{p}{(}\PYG{p}{)}
\end{sphinxVerbatim}

returns \sphinxcode{True} if the \sphinxstyleemphasis{RED} LED is off.

This module also provides the class {\hyperref[\detokenize{index:rcpy.led.Blink}]{\sphinxcrossref{\sphinxcode{rcpy.led.Blink}}}} to
handle LED blinking. It spawns a thread that will keep LEDs blinking
with a given period. For example:

\begin{sphinxVerbatim}[commandchars=\\\{\}]
\PYG{n}{blink} \PYG{o}{=} \PYG{n}{Blink}\PYG{p}{(}\PYG{n}{led}\PYG{o}{.}\PYG{n}{red}\PYG{p}{,} \PYG{o}{.}\PYG{l+m+mi}{5}\PYG{p}{)}
\PYG{n}{blink}\PYG{o}{.}\PYG{n}{start}\PYG{p}{(}\PYG{p}{)}
\end{sphinxVerbatim}

starts blinking the \sphinxstyleemphasis{RED} LED every 0.5 seconds. One can also
instantiate an {\hyperref[\detokenize{index:rcpy.led.Blink}]{\sphinxcrossref{\sphinxcode{rcpy.led.Blink}}}} object by calling
{\hyperref[\detokenize{index:rcpy.led.LED.blink}]{\sphinxcrossref{\sphinxcode{rcpy.led.LED.blink()}}}} as in:

\begin{sphinxVerbatim}[commandchars=\\\{\}]
\PYG{n}{blink} \PYG{o}{=} \PYG{n}{led}\PYG{o}{.}\PYG{n}{red}\PYG{o}{.}\PYG{n}{Blink}\PYG{p}{(}\PYG{o}{.}\PYG{l+m+mi}{5}\PYG{p}{)}
\PYG{n}{blink}\PYG{o}{.}\PYG{n}{start}\PYG{p}{(}\PYG{p}{)}
\end{sphinxVerbatim}

which produces the same result. One can stop or resume blinking by calling {\hyperref[\detokenize{index:rcpy.led.Blink.toggle}]{\sphinxcrossref{\sphinxcode{rcpy.led.Blink.toggle()}}}} as in:

\begin{sphinxVerbatim}[commandchars=\\\{\}]
\PYG{n}{blink}\PYG{o}{.}\PYG{n}{toggle}\PYG{p}{(}\PYG{p}{)}
\end{sphinxVerbatim}

or call:

\begin{sphinxVerbatim}[commandchars=\\\{\}]
\PYG{n}{blink}\PYG{o}{.}\PYG{n}{stop}\PYG{p}{(}\PYG{p}{)}
\end{sphinxVerbatim}

to permanently stop the blinking thread.


\section{Constants}
\label{\detokenize{index:id5}}\index{ON (in module rcpy.led)}

\begin{fulllineitems}
\phantomsection\label{\detokenize{index:rcpy.led.ON}}\pysigline{\sphinxcode{rcpy.led.}\sphinxbfcode{ON}}
State of an on LED; equal to {\hyperref[\detokenize{index:rcpy.gpio.HIGH}]{\sphinxcrossref{\sphinxcode{rcpy.gpio.HIGH}}}}.

\end{fulllineitems}

\index{OFF (in module rcpy.led)}

\begin{fulllineitems}
\phantomsection\label{\detokenize{index:rcpy.led.OFF}}\pysigline{\sphinxcode{rcpy.led.}\sphinxbfcode{OFF}}
State of an off led; equal to {\hyperref[\detokenize{index:rcpy.gpio.LOW}]{\sphinxcrossref{\sphinxcode{rcpy.gpio.LOW}}}}.

\end{fulllineitems}

\index{red (in module rcpy.led)}

\begin{fulllineitems}
\phantomsection\label{\detokenize{index:rcpy.led.red}}\pysigline{\sphinxcode{rcpy.led.}\sphinxbfcode{red}}
{\hyperref[\detokenize{index:rcpy.led.LED}]{\sphinxcrossref{\sphinxcode{rcpy.led.LED}}}} representing the Robotics Cape \sphinxstyleemphasis{RED} LED.

\end{fulllineitems}

\index{green (in module rcpy.led)}

\begin{fulllineitems}
\phantomsection\label{\detokenize{index:rcpy.led.green}}\pysigline{\sphinxcode{rcpy.led.}\sphinxbfcode{green}}
{\hyperref[\detokenize{index:rcpy.led.LED}]{\sphinxcrossref{\sphinxcode{rcpy.led.LED}}}} representing the Robotics Cape \sphinxstyleemphasis{GREEN} LED.

\end{fulllineitems}



\section{Classes}
\label{\detokenize{index:id6}}\index{LED (class in rcpy.led)}

\begin{fulllineitems}
\phantomsection\label{\detokenize{index:rcpy.led.LED}}\pysiglinewithargsret{\sphinxstrong{class }\sphinxcode{rcpy.led.}\sphinxbfcode{LED}}{\emph{output}, \emph{state = rcpy.led.OFF}}{}~\begin{quote}\begin{description}
\item[{Bases}] \leavevmode
\sphinxcode{rcpy.gpio.Output}

\item[{Parameters}] \leavevmode\begin{itemize}
\item {} 
\sphinxstyleliteralstrong{output} \textendash{} GPIO pin

\item {} 
\sphinxstyleliteralstrong{state} \textendash{} initial LED state

\end{itemize}

\end{description}\end{quote}

{\hyperref[\detokenize{index:rcpy.led.LED}]{\sphinxcrossref{\sphinxcode{rcpy.led.LED}}}} represents LEDs in the Robotics Cape or Beaglebone Blue.
\index{is\_on() (rcpy.led.LED method)}

\begin{fulllineitems}
\phantomsection\label{\detokenize{index:rcpy.led.LED.is_on}}\pysiglinewithargsret{\sphinxbfcode{is\_on}}{}{}~\begin{quote}\begin{description}
\item[{Returns}] \leavevmode
\sphinxcode{True} if LED is on and \sphinxcode{False} if LED is off

\end{description}\end{quote}

\end{fulllineitems}

\index{is\_off() (rcpy.led.LED method)}

\begin{fulllineitems}
\phantomsection\label{\detokenize{index:rcpy.led.LED.is_off}}\pysiglinewithargsret{\sphinxbfcode{is\_off}}{}{}~\begin{quote}\begin{description}
\item[{Returns}] \leavevmode
\sphinxcode{True} if LED is off and \sphinxcode{False} if LED is on

\end{description}\end{quote}

\end{fulllineitems}

\index{on() (rcpy.led.LED method)}

\begin{fulllineitems}
\phantomsection\label{\detokenize{index:rcpy.led.LED.on}}\pysiglinewithargsret{\sphinxbfcode{on}}{}{}
Change LED state to \sphinxcode{rcpy.LED.ON}.

\end{fulllineitems}

\index{off() (rcpy.led.LED method)}

\begin{fulllineitems}
\phantomsection\label{\detokenize{index:rcpy.led.LED.off}}\pysiglinewithargsret{\sphinxbfcode{off}}{}{}
Change LED state to \sphinxcode{rcpy.LED.OFF}.

\end{fulllineitems}

\index{toggle() (rcpy.led.LED method)}

\begin{fulllineitems}
\phantomsection\label{\detokenize{index:rcpy.led.LED.toggle}}\pysiglinewithargsret{\sphinxbfcode{toggle}}{}{}
Toggle current LED state.

\end{fulllineitems}

\index{blink() (rcpy.led.LED method)}

\begin{fulllineitems}
\phantomsection\label{\detokenize{index:rcpy.led.LED.blink}}\pysiglinewithargsret{\sphinxbfcode{blink}}{\emph{period}}{}~\begin{quote}\begin{description}
\item[{Parameters}] \leavevmode
\sphinxstyleliteralstrong{period} (\sphinxstyleliteralemphasis{float}) \textendash{} period of blinking

\item[{Returns}] \leavevmode
an instance of {\hyperref[\detokenize{index:rcpy.led.Blink}]{\sphinxcrossref{\sphinxcode{rcpy.led.Blink}}}}.

\end{description}\end{quote}

Blinks LED with a period of \sphinxtitleref{period} seconds.

\end{fulllineitems}


\end{fulllineitems}

\index{Blink (class in rcpy.led)}

\begin{fulllineitems}
\phantomsection\label{\detokenize{index:rcpy.led.Blink}}\pysiglinewithargsret{\sphinxstrong{class }\sphinxcode{rcpy.led.}\sphinxbfcode{Blink}}{\emph{led}, \emph{period}}{}~\begin{quote}\begin{description}
\item[{Bases}] \leavevmode
threading.Thread

\end{description}\end{quote}
\index{set\_period() (rcpy.led.Blink method)}

\begin{fulllineitems}
\phantomsection\label{\detokenize{index:rcpy.led.Blink.set_period}}\pysiglinewithargsret{\sphinxbfcode{set\_period}}{\emph{period}}{}~\begin{quote}\begin{description}
\item[{Parameters}] \leavevmode
\sphinxstyleliteralstrong{period} (\sphinxstyleliteralemphasis{float}) \textendash{} period of blinking

\end{description}\end{quote}

Set blinking period.

\end{fulllineitems}

\index{toggle() (rcpy.led.Blink method)}

\begin{fulllineitems}
\phantomsection\label{\detokenize{index:rcpy.led.Blink.toggle}}\pysiglinewithargsret{\sphinxbfcode{toggle}}{}{}
Toggle blinking on and off. Call toggle again to resume or stop blinking.

\end{fulllineitems}

\index{start() (rcpy.led.Blink method)}

\begin{fulllineitems}
\phantomsection\label{\detokenize{index:rcpy.led.Blink.start}}\pysiglinewithargsret{\sphinxbfcode{start}}{}{}
Start the blinking thread.

\end{fulllineitems}

\index{stop() (rcpy.led.Blink method)}

\begin{fulllineitems}
\phantomsection\label{\detokenize{index:rcpy.led.Blink.stop}}\pysiglinewithargsret{\sphinxbfcode{stop}}{}{}
Stop the blinking thread. Blinking cannot resume after calling {\hyperref[\detokenize{index:rcpy.led.Blink.stop}]{\sphinxcrossref{\sphinxcode{rcpy.led.Blink.stop()}}}}.

\end{fulllineitems}


\end{fulllineitems}



\chapter{Module \sphinxtitleref{rcpy.encoder}}
\label{\detokenize{index:module-rcpy-encoder}}\label{\detokenize{index:rcpy-encoder}}

\chapter{Module \sphinxtitleref{rcpy.mpu9250}}
\label{\detokenize{index:module-rcpy-mpu9250}}\label{\detokenize{index:rcpy-mpy9250}}

\chapter{Module \sphinxtitleref{rcpy.motor}}
\label{\detokenize{index:rcpy-motor}}\label{\detokenize{index:module-rcpy-motor}}

\chapter{Indices and tables}
\label{\detokenize{index:indices-and-tables}}\begin{itemize}
\item {} 
\DUrole{xref,std,std-ref}{genindex}

\item {} 
\DUrole{xref,std,std-ref}{modindex}

\item {} 
\DUrole{xref,std,std-ref}{search}

\end{itemize}


\renewcommand{\indexname}{Python Module Index}
\begin{sphinxtheindex}
\def\bigletter#1{{\Large\sffamily#1}\nopagebreak\vspace{1mm}}
\bigletter{r}
\item {\sphinxstyleindexentry{rcpy}}\sphinxstyleindexpageref{index:\detokenize{module-rcpy}}
\item {\sphinxstyleindexentry{rcpy.button}}\sphinxstyleindexpageref{index:\detokenize{module-rcpy.button}}
\item {\sphinxstyleindexentry{rcpy.gpio}}\sphinxstyleindexpageref{index:\detokenize{module-rcpy.gpio}}
\item {\sphinxstyleindexentry{rcpy.led}}\sphinxstyleindexpageref{index:\detokenize{module-rcpy.led}}
\end{sphinxtheindex}

\renewcommand{\indexname}{Index}
\printindex
\end{document}